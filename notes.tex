\documentclass[twoside]{article}
\usepackage[b4paper, margin=14.82mm, includehead]{geometry}
\usepackage{amsmath}
\usepackage{changepage}
\usepackage{enumitem}
\usepackage{fancyhdr}
\usepackage{fontspec}
\usepackage{graphicx}
\usepackage{hanging}
\usepackage{multicol}
\usepackage{musixtex}
\usepackage{wasysym}
\setmainfont[
 BoldFont={[ACADEMICO-BOLD.otf]}, 
 ItalicFont={[ACADEMICO-ITALIC.otf]},
 BoldItalicFont={[ACADEMICO-BOLDITALIC.otf]}
 ]{[ACADEMICO-REGULAR.otf]}
\setlength{\columnsep}{1cm}

\pagestyle{fancy}
\renewcommand{\headrulewidth}{0pt}
\fancyhf{}%clear all headers and footers
\fancyhead[LE]{\fontsize{8pt}{10pt}\selectfont \slshape\rightmark}
\fancyhead[RO]{\fontsize{8pt}{10pt}\selectfont \slshape\leftmark}
\fancyhead[LE,RO]{\thepage}
\setcounter{page}{89}

\newcommand\dynmark[1]{\scalebox{0.9}{#1}{\kern1pt}}
\newcommand{\nobarfrac}{\genfrac{}{}{0pt}{}}

\begin{document}
\begin{center}
\underline{\huge{Editorial Commentary}}
\end{center}

This edition is based on 3 sources: an undated and unsigned manuscript (FS0), a copyist's manuscript published by Heugel in 1928 (FS1), and a reduction for harp and piano published by Heugel in 1928 (RED). It is known in addition that a holograph manuscript from 1927 exists, but it is not available to the editor. The differences between the sources suggest the following chronology from earliest to latest: FS0, FS1, RED. Significant differences and ambiguities are listed below. More extensive notes are available in the source repository (particularly related to dynamics).

\begin{hangparas}{15pt}{1}
\begin{multicols}{2}

\underline{General}

Trumpet initially labeled ``TROMPETTE bouché'' in FS0 and FS1. The only mute indication is ``sourdine'' at II.7 in FS0. This has been interpreted as the trumpet muted throughout the whole piece.

``rit.'' (presumably ``ritardando'' but unclear), ``riten.,'' and ``rall.'' are used inconsistently across the sources.

This edition adds dotted ties to and extends the trill line over the presumed whole duration of trilled notes. In the original sources, the trill is usually written at each measure with no ties connecting the constituent notes, making rearticulation ambiguous.

This edition writes out the first instances of presumed measured tremolos. (No measured tremolos are written out in the sources.)

\underline{Movement I}

6--11, Harp: Ossia as in FS0 and FS1.

9, Harp: ♯ is penciled in on the C\textsubscript{3} only in FS0 and is missing in FS1. C(♮)\textsubscript{3} would fit the strings but is impossible to play given the upper staff, while C♯ is 2 octaves below the upper staff and the woodwinds.

11, Harp: A\textsubscript{3} (unmarked) in FS0. A♭\textsubscript{3} in FS1. Not present in RED. A♭ is more likely based on strings, but is impossible to play given the upper staff. Changing to G♯.

16, Horns: FS0 has ``con sordina'' penciled in. FS1 has explicit ``ouvert.''

31, Harp: Final 16\textsuperscript{th} is A\textsubscript{4} in FS0 and FS1, but see measure 8.

32, Harp: C\textsubscript{3} (unmarked) only in FS0. See measure 9.

36--38, Harp: Lower staff E\textsubscript{5}s are Ds in FS0 and FS1. These seem likely to be errors despite a D♮ pedal mark penciled in in FS0.

39, Violin II.2: This ``pizz.'' is likely intended to start on the off beat. It is written over the on-beat in FS1 and FS0, perhaps for room in FS0 and erroneously transferred to FS1.

43--45, Violins: This edition follows FS1. In FS0, the Vln. II upper divisi is given to Vln. I, and the Vln. II lower divisi is played by Vln. II in unison.

51, Viola: ``pizz.'' found only in FS1, which is likely a misreading of the ``pizz.'' written below the Vln. II staff. In addition, the first note is D\textsubscript{4} instead of E\textsubscript{4}, likely an error.

52, Harp: Second to last 16\textsuperscript{th} is G♯\textsubscript{5} in RED, A\textsubscript{5} in FS0 and FS1. See Flute 2 in measure 56.

53, Harp: Glissando lasts for a quarter note instead of an eighth note in FS0 and FS1. In addition, the final G♯\textsubscript{6} is written as part of the grace-note beam.

56, Flute 2: Second to last 16\textsuperscript{th} is G♯\textsubscript{5} in accompaniment of RED, A\textsubscript{5} in FS0 and FS1. See Harp in measure 52.

56--57, Harp: This section not present in FS0. Upper staff is marked \textit{\textbf{8\textsuperscript{va}}} in FS1.

58--59, Harp: Both staves an octave higher in FS0 and FS1.

65, Harp: G\textsubscript{2} of RED is F\textsubscript{2} in FS0 and FS1.

66, Harp: D♭ pedal change in FS0 and FS1. No change in RED.

67, Harp: Only FS1 marks the 3-note beam groups as triplets. Beat 2 is placed approximately between the first C\textsubscript{4} and E♭\textsubscript{4} in the sources.

72: ``Poco meno mosso \quarternote\ = 72'' only found in FS0.

72, Harp: Last note lower octave only in FS0 and FS1.

76: ``rit...'' present only in FS1.

90: ``poco animando...'' until ``Tempo 1\textsuperscript{o}'' at measure 94 only in FS0. See also note at measure 72.

94--107, Harp: Spelling of the accidentals has been rewritten to improve pedal changes and sonority. The original spelling is included in an appendix.

98--99, Harp: 16\textsuperscript{th} note E\textsubscript{3}s are F\textsubscript{3}s in FS1, likely errors.

106, Harp: The Nicanor Zabaleta recording, the only one made during Tailleferre's lifetime, has B♮\textsubscript{3}s. Reasoning not evident in the sources.

120: ``Pas trop vite'' only present in FS0 (ending ``pressez un peu'').

120--135, Harp: Ossias as in FS0 and FS1.

133, Flutes: A\textsubscript{5}s should perhaps be ♭ to match the A♭ of the Harp in RED. FS0 and FS1 have A(♮).

148--167, Harp: Cadenza in RED vs. FS0 and FS1 have significant differences. The cadenza as found in FS0 and FS1 is included in an appendix. The measure numbering in this edition follows FS0 and FS1.

173--174, Harp: Ossia as in FS0 and FS1. One of D\textsubscript{3} vs. C\textsubscript{3} is perhaps an error.

179--182, Cello: Trills crossed out in FS0.

183: Originally written as an explicit change to $\substack{3\\4}$. Several instruments are still playing $\substack{6\\8}$, so this edition writes ``battre la mesure à $\substack{3\\4}$'' instead.

183--186, Violin I: 4-note group slurs in FS1. 6-note group slurs in FS0. Originally beamed in groups of 6 notes in FS1, but it has been edited into 4-note beams.

201: Sim. 183.

201--208, Harp: Ossia as in FS0.

\underline{Movement II}

3--18, Harp: Ossias as in FS0.

7, Violin I.2: Harmonic is impossible to play and intended octave is unclear. Following measure 33, it has been replaced with the octave below divisi 1.

14, Flute 1: This measure is only in FS1, added as an extra staff.

35, Trumpet: Rhythm in RED is 16\textsuperscript{th} + dotted 8\textsuperscript{th}.

\underline{Movement III}

28, Harp: C\textsubscript{5} not present in FS0 or FS1. See measure 202.

31, Timpani: C\textsubscript{3} in FS1. D\textsubscript{3} in FS0. See contrabass, but the change may be to make the part easier with 2 drums.

42--43, Horns: These measures are played by Horn 1 instead of Horn 2 in FS0.

47, Harp: Ossia as in FS0 and FS1.

67--68, Violin II.2: Tremolos continue in FS0.

67--68, Cello: Ossia as in FS0.

69, Viola.2: C\textsubscript{4} changed to D\textsubscript{4} in FS0, matching upper divisi.

72--80, Harp: No E♯s in FS0.

92, Harp: No pedal change indications, but the harmony of the following fugue implies G major.

138--140, Harp: Lower staff only in FS0 and FS1.

140, Cello: Divisi only in FS0.

148--149, Cello: Triple beam tremolos in FS0. Double beam tremolos in FS1, implying measured tremolos.

152--153, Horn 1: These measures tacet in FS0.

161, Harp: Ossia as in FS0.

174, Harp: Lower staff beat 2 only in FS0. The tempo indications are inconsistent across the sources. RED has ``poco rit.'' starting at the C\textsubscript{4} of the glissando. FS0 has ``assez lent'' written at the start of the glissando. FS1 has no tempo indications.

188--189, Harp: The lower octave in the second beat of 188 and the first beat of 189 is only present in RED.

205, Harp: The A\textsubscript{3} in beat 2 is a G\textsubscript{3} in FS1 and RED. These are likely errors, as A\textsubscript{3} matches the Cellos.

216, Flute: Ossia as in FS0.

225--231, Piccolo: This section not present in FS0.

243, Horn 1: Last note is A\textsubscript{4} in FS1, likely an error given shape of string patterns.

249--251, Harp: E♭\textsubscript{3}s only present in RED.

257, Violin II: Ossia as in FS0.

259--260, Woodwinds: Ossias as in FS0.

262, Timpani: G\textsubscript{2} in FS0. F\textsubscript{2} in FS1. Should perhaps be G♭\textsubscript{2} to match other instruments.

266, Harp: B♭\textsubscript{4} is omitted from the upper cluster in FS0 and FS1. E\textsubscript{3} is omitted from the lower cluster in RED, while D\textsubscript{3} is omitted in FS0 and FS1. It is unrealistic for any of these notes to be excluded.

\end{multicols}

\end{hangparas}

\end{document}
