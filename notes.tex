\documentclass[twoside]{article}
\usepackage[b4paper, margin=14.82mm, includehead]{geometry}
\usepackage{amsmath}
\usepackage{changepage}
\usepackage{enumitem}
\usepackage{fancyhdr}
\usepackage{fontspec}
\usepackage{graphicx}
\usepackage{hanging}
\usepackage{harmony}
\usepackage{multicol}
\usepackage{musixtex}
\usepackage{wasysym}
\setmainfont[
 BoldFont={[ACADEMICO-BOLD.otf]}, 
 ItalicFont={[ACADEMICO-ITALIC.otf]},
 BoldItalicFont={[ACADEMICO-BOLDITALIC.otf]}
 ]{[ACADEMICO-REGULAR.otf]}
\setlength{\columnsep}{1cm}

\pagestyle{fancy}
\renewcommand{\headrulewidth}{0pt}
\fancyhf{}%clear all headers and footers
\fancyhead[LE]{\fontsize{8pt}{10pt}\selectfont \slshape\rightmark}
\fancyhead[RO]{\fontsize{8pt}{10pt}\selectfont \slshape\leftmark}
\fancyhead[LE,RO]{\thepage}
\setcounter{page}{89}

\newcommand\dynmark[1]{\scalebox{0.9}{#1}{\kern1pt}}
\newcommand{\nobarfrac}{\genfrac{}{}{0pt}{}}

\begin{document}
\begin{center}
\underline{\huge{Editorial Commentary}}
\end{center}

This edition is based on 3 sources: an undated and unsigned manuscript (FS0), a copyist's manuscript published by Heugel in 1928 (FS1), and a reduction for harp and piano published by Heugel in 1928 (RED). It is known in addition that a holograph manuscript from 1927 exists, but it is not available to the editor. The differences between the sources suggest the following chronology from earliest to latest: FS0, FS1, RED. Significant differences and ambiguities are listed below. More extensive notes are available in the source repository.

\begin{hangparas}{15pt}{1}
\begin{multicols}{2}

\underline{General}

Trumpet initially labeled ``TROMPETTE bouchée'' in FS0 and FS1, interpreted as the trumpet muted throughout the whole piece. The only mute indication is ``sourdine'' at II.7 in FS0.

``rit.'' (presumably ``ritardando'' but unclear), ``riten.,'' and ``rall.'' are used inconsistently across the sources.

This edition adds dashed ties to and extends the trill line over the presumed whole duration of trilled notes. In the sources, a trill sign is usually written at each measure with no ties connecting the constituent notes, making rearticulation ambiguous.

This edition writes out some presumed measured tremolos. (No measured tremolos are written out in the sources.)

\underline{Movement I}

6--11, Harp: Ossia as in FS0 and FS1.

9, Harp: ♯ is penciled in on the C\textsubscript{3} only in FS0 and is missing in FS1. C(♮)\textsubscript{3} would match the strings but is impossible to play given the upper staff, while C♯\textsubscript{3} is 2 octaves below the upper staff and the woodwinds.

11, Harp: A\textsubscript{3} (unmarked) in FS0. A♭\textsubscript{3} in FS1. Not found in RED. A♭ is more likely based on strings, but is impossible to play given the upper staff. Changing to G♯.

16, Horns: FS0 has ``con sordina'' penciled in. FS1 has ``ouvert.''

31, Harp: Final 16\textsuperscript{th} is A\textsubscript{4} in FS0 and FS1, but see measure 8.

32, Harp: C\textsubscript{3} (unmarked) only in FS0. See measure 9.

36--38, Harp: Lower staff E\textsubscript{5}s are Ds in FS0 and FS1. These seem likely to be errors despite a D♮ pedal mark penciled in in FS0.

39, Violin II.2: ``pizz.'' written at start of measure in FS0 and FS1.

43--45, Violins: This edition follows FS1. In FS0, the Vln. II upper divisi is given to Vln. I, and the Vln. II lower divisi is played by Vln. II in unison.

52, Harp; 56, Flute 2: Second to last 16\textsuperscript{th} is G♯\textsubscript{5} in RED, A\textsubscript{5} in FS0 and FS1.

53, Harp: Glissando extends a quarter instead of an eighth in FS0 and FS1, with the final G♯\textsubscript{6} as part of the grace-note beam.

56--57, Harp: This section not found in FS0. Upper staff is marked \textit{\textbf{8\textsuperscript{va}}} in FS1.

58--59, Harp: Both staves an octave higher in FS0 and FS1.

65, Harp: G\textsubscript{2} of RED is F\textsubscript{2} in FS0 and FS1.

66, Harp: D♭ pedal change in FS0 and FS1. No change in RED.

67, Harp: 3-note beam groups marked as triplets only in FS1.

72: ``Poco meno mosso \Vier\ = 72'' only found in FS0.

72, Harp: Last note lower octave only in FS0 and FS1.

76: ``rit...'' found only in FS1.

82, Harp: Harmonic on D\textsubscript{3} presumably missing in FS0 and FS1. This note is on beat 5 in RED (measure notated as $\substack{3\\4}$\,), likely an error, and on beat 4 in FS0 and FS1 (measure notated as $\substack{6\\8}$\,).

90: ``poco animando...'' until ``Tempo 1\textsuperscript{o}'' at measure 94 only in FS0. See also note at measure 72.

94--107, Harp: Spelling of the accidentals has been rewritten to improve pedaling and sonority. The original spelling is included in an appendix.

106, Harp: The Nicanor Zabaleta recording, the only one made during Tailleferre's lifetime, has B♮\textsubscript{3}s. Reasoning not evident in the sources.

120: ``Pas trop vite'' only found in FS0 (ending ``pressez un peu'').

120--135, Harp: Ossias as in FS0 and FS1.

133, Flutes: A\textsubscript{5}s should perhaps be ♭ to match the A♭ of the Harp in RED. FS0 and FS1 have A(♮).

148--167, Harp: The cadenza as found in FS0 and FS1 is included in an appendix, as it is significantly different. The measure numbering in this edition follows FS0 and FS1.

173--174, Harp: Ossia as in FS0 and FS1. One of D\textsubscript{3} vs. C\textsubscript{3} is perhaps an error.

179--182, Cello: Trills crossed out in FS0.

183: Originally written as an explicit change to $\substack{3\\4}$. Several instruments are still playing $\substack{6\\8}$, so this edition writes ``battre la mesure à $\substack{3\\4}$'' instead.

183--186, Violin I: Slurred in 4-note groups in FS1, 6-note groups in FS0. Originally beamed in groups of 6 notes in FS1, but it has been edited into 4-note beams.

188, Piccolo: The second 16\textsuperscript{th} is E\textsubscript{6} in FS0 and FS1 (not found in RED). Changed to D\textsubscript{6} to match Harp pattern and Clarinet.

191, Piccolo: Trill extends over this quarter note in FS0.

193: ``\Vier\ = 92'' only in FS0; it and ``Poco'' are crossed out.

201: Time signature sim. 183.

201--208, Harp: Ossia as in FS0.

\underline{Movement II}

3--18, Harp: Ossias as in FS0.

7, Violin I.2: Natural harmonic is impossible to play and intended octave is unclear. Following measure 33, the octave below the upper divisi is suggested.

14, Flute 1: This measure is only in FS1, added as an extra staff.

19, Harp: The (quasi) glissando is awkward to play as written. E♮ and B♮ is suggested to maintain the same pedaling as the following measures, or B♯ to be enharmonic with C♮. (However, no explicit pedaling is given at 20.)

35, Trumpet: Rhythm in RED is \SechBr \Vier\,\Pu\ .

\underline{Movement III}

28, Harp: C\textsubscript{5} not found in FS0 or FS1. See measure 202.

31, Timpani: C\textsubscript{3} in FS1. D\textsubscript{3} in FS0. See Contrabass, but the change may be to make the part easier with 2 drums.

42--43, Horns: These measures are played by Horn 1 instead of Horn 2 in FS0.

47, Harp: Ossia as in FS0 and FS1.

48, Harp: No pedaling given, but G major matches orchestra. 

67--68, Violin II.2: Tremolos continue in FS0.

67--68, Cello: Ossia as in FS0.

69, Viola.2: C\textsubscript{4} changed to D\textsubscript{4} in FS0, matching upper divisi.

72--80, Harp: No E♯\textsubscript{4}s in FS0.

86, Flute 1, Clarinet: The final concert E\textsubscript{5} has no accidental in FS0 and FS1, and RED has an explicit E♮. This exception in the descending patterns seems unlikely; E♭ is more likely. 

92, Harp: No pedaling given. Following fugue suggests G major.

117, Cello: First 16\textsuperscript{th} is B♮\textsubscript{3} in FS0, A(♮)\textsubscript{3} in FS1. Not found in RED. 

134, Flute 1: Second beat \SechBR \Vier \AchtBL\ \ in FS0. \AchtBR \Vier \SechBL\ \ in FS1. Not found in RED. Harmony of FS0 seems more likely.

138--140, Harp: Lower staff only in FS0 and FS1.

140, Cello: Divisi only in FS0.

148--149, Cello: Triple beam tremolos in FS0. Double beam tremolos in FS1, implying measured tremolos.

152--153, Horn 1: These measures tacet in FS0.

161, Harp: Ossia as in FS0.

174, Harp: Lower staff beat 2 only in FS0. RED has ``poco rit.'' starting at the C\textsubscript{4} of the glissando. FS0 has ``assez lent'' written at the start of the glissando. FS1 has no tempo indications.

188--189, Harp: The lower octave in the second beat of 188 and the first beat of 189 is only found in RED.

205, Harp: The A\textsubscript{3} in beat 2 is a G\textsubscript{3} in FS1 and RED. These are likely errors, as A\textsubscript{3} matches the Cellos.

216, Flute: Ossia as in FS0.

225--231, Piccolo: This section not found in FS0.

249--251, Harp: E♭\textsubscript{3}s only found in RED.

257, Violin II: Ossia as in FS0.

260, Piccolo, Flute 1: In FS0, this descending run from G is written at 259 (likely an error) for the Piccolo, and copied by a second hand to Flute 1.

260, Clarinet: This measure found only in FS0.

262, Timpani: G\textsubscript{2} in FS0. F\textsubscript{2} in FS1. Should perhaps be G♭\textsubscript{2} to match other instruments.

266, Harp: B♭\textsubscript{4} is omitted from the upper cluster in FS0 and FS1. E\textsubscript{3} is omitted from the lower cluster in RED, while D\textsubscript{3} is omitted in FS0 and FS1. It is unrealistic for any of these notes to be excluded.

\end{multicols}

\end{hangparas}

\end{document}
